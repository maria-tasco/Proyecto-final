\documentclass{article}
\usepackage[utf8]{inputenc}
\usepackage[spanish]{babel}
\usepackage{listings}
\usepackage{graphicx}
\graphicspath{ {images/} }
\usepackage{cite}

\begin{document}

\begin{titlepage}
    \begin{center}
        \vspace*{1cm}
            
        \Huge
        \textbf{Proyecto Final}
            
        \vspace{0.5cm}
        \LARGE
        Ideación
            
        \vspace{1.5cm}
            
        \textbf{Maria Fernanda Tasco Alquichire}
            
        \vfill
            
        \vspace{0.8cm}
            
        \Large
        Despartamento de Ingeniería Electrónica y Telecomunicaciones\\
        Universidad de Antioquia\\
        Medellín\\
        Marzo de 2021
            
    \end{center}
\end{titlepage}

\tableofcontents
\newpage
\section{Sección introductoria}\label{intro}
El proyecto final de la materia de informatica 2 consta del diseño y la creación de un video juego, usando C++ en QT creator. En este documento se presenta la etapa del diseño plasmando las ideas \cite{ideas}, tematica, movimientos posibles que nos dará una idea y será una base para el desarrollo del video juego.

\section{Sección de contenido} \label{contenido}

\subsection{Espacio del juego}
El mensajero (The Harbinger). llevar un mensaje a los 3 pueblos elegidos (Romans, Galatians, Colossians) para lo cual tendrá que pasar por muchas travesias.
Cada pueblo tendrá su adversario con sus caracteristicas para impedir llevar el mensaje.


\subsection{Las dimenciones mecanicas: Reglas del juego}
Al iniciar el juego se podrá elegir el persona entre mujer u hombre.

\vspace{0.1cm}
En cada pueblo de darán algunas virtudes (movimientos) al jugador que le ayudará a enfrentar a su adversario y llevar el mensaje.

\vspace{0.1cm}
Si el adversario lográ quitarle el mensaje, el mensajero tendrá 3 vidas u oportunidades para volver a intentarlo, de lo contrario el mensajero volverá al pueblo donde inicio.

\vspace{0.1cm}
En el primer pueblo (Romans) El adversario estará en un lugar del pueblo lanzando piedras desde distintos lugares aleatorios con un tiro parabolico, el mensajero tendrá que saltar esquivando las piedras para no ser golpeado ya que al ser golpeado perderá 1 de 3 vidas u oportunidades para llevar el mensaje a ese pueblo.

\vspace{0.1cm}
En el segundo pueblo (Galatians), cada vez que el adversario lanze la llama de la quietud al mensajero, este tendrá 3 herramientas que le ayudarán a liberarse, la idea es poder entregar el mensaje sin ser tocado por esa llama.

\vspace{0.1cm}
En el tercer pueblo (Colossians) constará de superficies o pequeñas plataformas en movimiento en las cuales el mensajero tendrá que subir hasta llegar a la cima donde entregará el último mensaje; pero ahora tendrá a dos adversarios (los mismos de los pueblos anteriores) atacando e impidiendo llegar a la cima. El mensajero tendrá ayudas para poder llegar: como impulso para saltar más alto, y protección.
\subsection{El juego es movimiento}
Algunos de los mivimientos planteados son: saltar, agarrar o quitar, tiro parablico, quietud.
Algunas magnitudes como fuerza, impulso (magnitud vectorial que tiene la dirección y el sentido de la fuerza que lo produce), medidas como angulo para el tiro parabolico.


\bibliographystyle{IEEEtran}
\bibliography{references}

\end{document}
